\section{Introducción}
En este texto hablaremos sobre qué es una curva \emph{suave} en $\realR^{2}$ y $\realR^{3}$. Cómo calcular su vector tangente en cualquier punto de la curva y así determinar
su espacio tangente asociado.

Llevaremos nuestra primera definición de curva a una forma más general y veremos como calcular el espacio tangente con esta nueva definición. Para esto veremos el teorema
de la función implícita (funciones $F(x,y)=0$).

Terminaremos con una breve introdución al concepto de variedad de dimension $n$ y su espacio tangente.
\section{Justificación}
En los cursos de Cálculo se define el concepto de superficie y el espacio tangente asociado a la superficie. Una superficie es un caso particular de un objeto más abstracto llamado variedad. Las variedades son objetos que se estudian en distintas áreas de la matemática: Topología, Análisis Matemático, Geometría Diferencial, Geometría Algebraica, etc. Es por este motivo que resulta de gran interés para cualquier estudiante de la carrera en matemáticas. En esta exposición realizamos un estudio de las variedades de dimensión dos y tres las cuales generalizan a las superficies y detallaremos la manera de calcular el espacio tangente.

\section{Objetivos}
Definir el concépto general de curva en $\realR^{2}$ y $\realR^{3}$ y su espacio tangente. Introducir brevemente el concepto de variedad y su espacio tangente.
%*****************************************

