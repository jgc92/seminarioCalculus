\section{Introducción}
En este texto hablaremos sobre qué es una curva \emph{suave} en $\realR^{2}$ y $\realR^{n}$, cómo calcular su vector tangente en cualquier punto de la curva y así determinar
su espacio tangente asociado.

Llevaremos nuestra primera definición de curva a una forma más general y veremos como calcular el espacio tangente con esta nueva definición. Para esto veremos el teorema
de la función implícita (funciones $F(x,y)=0$).

Terminaremos con una breve introdución al concepto de variedad de dimensi\'on $n$ y su espacio tangente.
\section{Justificación}
En los cursos de Cálculo se define el concepto de curva y el espacio tangente asociado a la curva. Una curva es un caso particular de un objeto más abstracto llamado subvariedad. Las subvariedades son objetos que se estudian en distintas áreas de la matemática: Topología, Análisis Matemático, Geometría Diferencial, Geometría Algebraica, etc. Es por este motivo que resulta de gran interés para cualquier estudiante de la carrera en matemáticas. En esta exposición realizamos un estudio de las subvariedades de dimensión 1 las cuales generalizan a las curvas y detallaremos la manera de calcular el espacio tangente.

\section{Objetivos}
Definir el concepto general de curva en $\realR^{2}$ y $\realR^{n}$ y su espacio tangente. Introducir brevemente el concepto de variedad y su espacio tangente. Demostrar que una variedad y su espacio tangente tienen la misma dimensi\'on en un punto $p$ de la variedad.
%*****************************************

