%************************************************
\chapter{Introducción}\label{ch:introduccion}
%************************************************
Cuando consideramos una función $f:\realR\mapsto\realR$, se tiene que $f$ es diferenciable en $a$ si la derivada de $f$ en $a$ existe, es decir $f'(a)$ existe. En este caso, el espacio tangente a $f$ en $a$ es una línea recta, esto es un espacio vectorial de dimensión 1. En general, si $f:\realR^{n}\mapsto\realR^{m}$, se tiene que $f$ es continua en $a$ si existe una matriz $T$ tal que

    $$\lim_{h \to 0} \| f(x+h) - f(x) - T(x) \cdot h \| = 0$$

Cuando $f$ es diferenciable, el espacio tangente está bien determinado y es un espacio vectorial de dimensión $n$.

Sin embargo cuando $f$ no es diferenciable en $a$, ¿Cómo determinamos el espacio tangente?, ¿Qué dimensión tiene?, ¿Que relación existe entre la dimensión del espacio tangente y la dimensión del espacio normal?.

Estas preguntas pueden ser extendidas a espacios mas generales como las variedades.
En el presente texto nos introduciremos al concepto de variedad y espacio tangente asociado a una variedad.


\section{Objetivos}

El objetivo de este texto es definir de manera general el concepto de superficie en $\realR^{3}$ y variedad en $\realR^{n}$.
Determinar el plano tangente de una superficie en $\realR^{3}$ (variedad en $\realR^{n}$). Determinar la dimensión del espacio tangente a una superficie (variedad) en un punto $p$. Definir el concepto de curvatura; con este estudio se darán algunos ejemplos de cálculo de curvatura y su aplicación a la Física.

\section{Justificación}

Es común ver en los estudiantes de cálculo clásico una deficiencia en el concepto de que es una superficie (variedad) y como tiene su propio cálculo diferencial estrictamente comparable con el cálculo familiar en el plano.\footnote{Mas adelante veremos como el último es consecuencia del primero.}

Esta exposición provee la noción de variedades diferenciables, la cual es indispensable en algunas ramas de las Matemáticas y sus aplicaciones basadas en el cálculo.

\section{Requisitos Previos y Notación}

Suponemos que el lector ha estudiado cálculo de funciones de una variable real, incluida la geometría analítica en el plano, asi como nociones de teoría matricial y álgebra lineal en especial homomorfismos. También suponemos que el lector esta familiarizado con funciones del cálculo elemental, como $sin(x)$, $cos(x)$, $e^{x}$ y $ln(x)$. En caso de no estar familiarizado vease el ápendice \{ incluir apendice \}.

Ahora enunciaremos las notaciones que usaremos, esto se puede leer rápidamente o saltarse y después volver si fuera necesario.

A la colección de numeros reales la denotaremos $\realR$. Cuando escribimos $a \in \realR$ nos referimos que $a$ es un elemento del conjunto $\realR$, es decir, que $a$ es un número real. Dados dos números reales $a$ y $b$ con $a < b$, podemos formar el intervalo cerrado $[a,b]$ formado por todos los $x$ tales que $a \le x \le b$, y el intervalo abierto $(a,b)$ formado por todos los $x$ tales que $a < x < b$.

Si escribimos $A \subset \realR$, nos referimos a que $A$ es un subconjunto de $\realR$. El símbolo $A \cup B$ significa la unión de $A$ y $B$, esto es, los elementos que están tanto en $A$ como en $B$. Escribimos $A \cap B$, que significa la intersección de $A$ y $B$, esto es, los elementos de $A$ y $B$ que están tanto en $A$ como en $B$. $A \ B$ lo usamos para denotar elemento de $A$ que no están en $B$.

Una función $f:A \mapsto B$ es una regla que asigna a cada $a \in A$ un elemento especifico $f(a)$ de $B$. Que la función $f$ mande $a$ a $f(a)$ se denota por $a \mapsto f(a)$.

\{Incluir imagenes de funciones e intervalos\} 
%*****************************************




