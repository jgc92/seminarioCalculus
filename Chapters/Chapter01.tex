%************************************************
\chapter{Vectores}\label{ch:vectores}
%************************************************

El concépto de vector es básico para el estudio de funciones de varias variables. Da una motivación
geométrica para todo lo que veremos. En esta sección platicaremos sobre algunas propiedades del vector
que necesitaremos para las secciones siguientes.

Una propiedad significativa de todos los enunciados y demostraciones de esta seccion es que aplican
facilmente para 2-dimensión, 3-dimensión y n-dimensión.

\section{Un punto en el espacio}

Sabemos que un número puede ser usado para representar un punto en una línea, siempre y cuando se
elija una unidad de longuitud.

Un par de números (un par ordenado) $(x,y)$ puede ser usado para representar un punto en un plano.

imagen

Ahora observemos que $(x,y,z)$ puede ser usado para representar un punto en el espacio, esto es un
punto en 3-dimensión, o 3-espacio. Simplemente agregamos un nuevo eje.

imagen


En vez de usar $x,y,z$ podemos usar $(x_{1},x_{2},x_{3})$. A la línea se le puede llamar 1-espacio y
al plano, naturalmente, 2-espacio.

Aunque no podemos dibujar un punto en 4-espacio, no hay nada que nos prevenga considerarlo, una cuarteta
de números

$$(x_{1},x_{2},x_{3},x_{4})$$

es un punto en 4-espacio. Una quinteta seria un punto en 5-espacio, si continuamos llegamos a la siguiente
definición.

\begin{definition}
    Un punto en n-espacio es una n-tupla ordenado de números
    $$ (x_{1},x_{2}, \ldots ,x_{n}) $$
    con $n$ un entero positivo.
\end{definition}

La mayoria de nuestros ejemplos serán para los casos $n=2$, $n=3$. Así podremos visualizarlos facilmente a
lo largo de este texto. Pero nuestras definiciones y demostraciones comunmente serán para n-espacio y así
cubrir ambos casos. Cabe mencionar que el caso $n=4$ ocurre en física.

La idea de tomar al tiempo como cuarta coordenada es vieja. Ya desde la \emph{Encyclopédie} de Diderot, en
el siglo XVIII, d'Alembert escribe en su artículo sobre dimensión:

\emph{La manera de considerar cantidades con mas de tres dimensiones es tan natural como los otros casos,
porque las letras algebraicas siempre pueden ser vistas como representación de números, ya sean racionales
o no. Antes mencione que no era posible concivir mas de tres dimensiones. Un caballero astuto de quien soy
íntimo cree lo contrario, que podria tomarse a la duración como la cuarta dimensión. Esta idea puede ser
discutida, pero para mí, tiene su mérito por el solo hecho de ser nueva.}

\begin{flushright}
    \emph{Encyplopédie, Vol.4 (1754), p. 1010}
\end{flushright}

%*****************************************
%*****************************************
%*****************************************
%*****************************************
%*****************************************




