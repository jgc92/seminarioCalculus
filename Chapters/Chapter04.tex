\chapter{1-variedad}
\section{Teorema de la funci\'on impl\'icita}
Este teorema es uno de aquellos que es dif\'icil de comprender por las
diferentes condiciones que se deben cumplir, del mismo modo, llega a 
varias conclusiones. Comenzaremos con un ejemplo.

\begin{example}
    Tenemos el c\'irculo unitario $x^{2} + y^{2} = 1$ con centro en el
    origen. Sabemos que  $x^{2} + y^{2} = 1$ no es una funci\'on ya que
    para cada $x$ en el dominio tenemos dos $f(x)$ en el rango, esto se
    puede ver claramente en la figura \ref{fig:unit-circle-1}. Pero si
    tomamos una vecindad de $x$ como en la figura \ref{fig:unit-circle-2}
    podemos definir la funci\'on $y = \sqrt{1-x^{2}}$ o $y = -\sqrt{1-x^{2}}$
    dependiendo de si $x$ esta por arriba o por abajo del eje-$x$. La
    pregunta ahora es ¿podemos hacer este mismo razonamiento para cualquier
    punto en el c\'irculo?. La respuesta es no. Si tomamos los puntos
    $(1,0)$ y $(-1,0)$ no podemos encontrar una vecindad lo suficientemente
    pequeña que cumpla que por cada $x$ tengamos un \'unico $f(x)$, ver
    figura \ref{fig:unit-cicle-3}.

    Hasta ahora tenemos que
    $$y = \sqrt{1-x^{2}} \text{,} \quad y = \sqrt{1-x^{2}}$$
    son dos funciones \emph{explicitas} que dan la misma relaci\'on que
    $x^{2} + y^{2} = 1$, de forma local para cada punto excepto para
    $(1,0)$ y $(-1,0)$.

\end{example}
