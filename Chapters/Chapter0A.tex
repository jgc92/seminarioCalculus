%********************************************************************
% Appendix
%*******************************************************
% If problems with the headers: get headings in appendix etc. right
%\markboth{\spacedlowsmallcaps{Appendix}}{\spacedlowsmallcaps{Appendix}}
\chapter{Algunas definiciones \'utiles}

\begin{definition}[Morfismo]
    \label{def:morphism}
    Es un mapeo que conserva la estructura de un objeto matem\'atico a otro. En este texto, vemos a los
    morfismos como funciones cuando van de conjunto a conjunto, y como transformaciones lineales cuando
    se define entre espacios vectoriales.
\end{definition}

\begin{definition}[Isomorfismo]
    \label{def:isomorphism}
    Es un morfismo que admite una inversa.
\end{definition}

\begin{theorem}
    Dos espacios vectoriales son isomorfos si y solo si tienen la misma dimensi\'on.
\end{theorem}

\begin{lemma}
    \label{lem:vector-space-dimension}
    Si dos espacios vectorial son isomorfos entonces tienen la misma dimensi\'on.
\end{lemma}

\begin{definition}[Difeomorfismo]
    \label{def:diffeomorphism}
    Es un isomorfismo diferenciable. Su inversa tambi\'en es diferenciable.
\end{definition}

